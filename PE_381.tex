\addcontentsline{toc}{chapter}{381 - (prime-k) factorial}
\chapter*{381 - (prime-k) factorial}

\index{Extended Euclidean} \index{inverse multiplicative} \index{modular arithmetic} \index{Wilson's theorem}

For a prime $p$ let $S(p) = \left ( \sum_{k=1}^5(p-k)! \right ) \bmod{p}$.\\

For example, if $p=7$,

\begin{align*}
    \sum_{k=1}^5(p-k)! &= (7-1)! + (7-2)! + (7-3)! + (7-4)! + (7-5)!\\
    &= 6! + 5! + 4! + 3! + 2!\\
    &= 720+120+24+6+2\\
    &= 872.
\end{align*}


As $872 \bmod{7}$ = 4, $S(7) = 4$.\\

It can be verified that $\sum_{p=5}^{99} S(p) = 480$.\\

Find $\sum_{p=5}^{10^8} S(p)$.

\section*{Solution}

By doing simple math and using modular arithmetic we get the $S(p)$ is defined by

\begin{align*}
    S(p) &= \left ( (p-5)! \bmod p \right ) \times 9 \\
    &= \left ( \left ( \frac{(p-1)!}{(p-1)(p-2)(p-3)(p-4)} \right ) \bmod p \right ) \times 9
\end{align*}

According to Wilson's Theorem \cite{wilson_theorem},

$$
(n-1)! \bmod n = -1 \bmod n
$$

When $n$ is prime.\\

Then, we only need to get rid of the denominator by using modular arithmetic. More precisely by using the inverse multiplicative. Suppose we have the following operation

$$
\frac{a}{b} \bmod m
$$

we can rewrite it as

$$
\left ( a \bmod m \right ) \left ( \frac{1}{b} \bmod m \right )
$$

The inverse multiplicative of $b$ is a number $k$, such that $(bk) \bmod m = 1$. Getting back to our problem, we are looking for an integer $k$ such that

$$
k(p-1)(p-2)(p-3)(p-4) \bmod p = 1
$$

Using modular arithmetic we have that

\begin{align*}
    (k(-1)(-2)(-3)(-4)) \bmod p &= 1 \\
    24k \bmod p &= 1
\end{align*}

Then 

\begin{align*}
    S(p) &= \left ( \frac{(p-1)!}{24} \bmod{p} \right ) \times 9 \\
    &= \left ( (p-1)! \frac{3}{8} \right ) \bmod p \\
    &= (-1 \bmod p) (3k \bmod p) \\
    &= (-3 \bmod p) (k \bmod p)\\
    &= (p-3)(k \bmod p)
\end{align*}

where $8k \bmod p = 1$\\

The value of $k$ can be found using the \textit{Extended Euclidean Algorithm}, which finds the inverse multiplicative of a number $a \bmod n$, only when $a$ and $m$ are coprimes. Since for our problem $m$ is prime, we can use it.\\

The \textit{Extended Euclidean Algorithm} find the values of $x$ and $y$ that solve the following equation:

$$
ax + by = 1
$$

Using $a=8$ and $b=p$ we have

$$
8x + py = 1
$$

Obtaining the modulus in both sides of the equation we get

\begin{align*}
    (8x + py) \bmod p &= 1 \\
    8x \bmod p &= 1
\end{align*}

which is what we are looking for. So basically we must calculate the value of $x$ using the \textit{Extended Euclidean Algorithm} and add the value of $p$ to avoid negative values. Then

$$
(x,y) = \mbox{ExtendedEuclidean}(8,p)
$$

and

$$
k = \left (x + p \right ) \bmod p
$$

Finally the value of $S(p)$ is given by

$$
S(p) = \left ( (p-3) \times k \right ) \bmod p
$$