\addcontentsline{toc}{chapter}{101 - Optimum polynomial}
\chapter*{101 - Optimum polynomial}

\index{least squares}
If we are presented with the first k terms of a sequence it is impossible to say with certainty the value of the next term, as there are infinitely many polynomial functions that can model the sequence.\\

As an example, let us consider the sequence of cube numbers. This is defined by the generating function, 
$u_n = n^3: 1, 8, 27, 64, 125, 216, \cdots$\\

Suppose we were only given the first two terms of this sequence. Working on the principle that ''simple is best'' we should assume a linear relationship and predict the next term to be 15 (common difference 7). Even if we were presented with the first three terms, by the same principle of simplicity, a quadratic relationship should be assumed.\\

We shall define $OP(k, n)$ to be the $n^{th}$ term of the optimum polynomial generating function for the first $k$ terms of a sequence. It should be clear that $OP(k, n)$ will accurately generate the terms of the sequence for $n \leq k$, and potentially the first incorrect term (FIT) will be $OP(k, k+1)$; in which case we shall call it a bad $OP (BOP)$.\\

As a basis, if we were only given the first term of sequence, it would be most sensible to assume constancy; that is, for $n \geq 2, OP(1, n) = u_1$.\\

Hence we obtain the following OPs for the cubic sequence:

\begin{align*}  
    OP(1, n) &= 1 \hphantom{4} (1, \textcolor{red}{1}, 1, 1, \ldots) \\
    OP(2, n) &= 7n-6 \hphantom{4} (1, 8, \textcolor{red}{15}, \ldots) \\
    OP(3, n) &= 6n^2-11n+6 \hphantom{4} (1, 8, 27, \textcolor{red}{58}, \ldots) \\
    OP(4, n) &= n^3 \hphantom{4} (1, 8, 27, 64, 125, \ldots) \\
\end{align*}

Clearly no BOPs exist for $k \geq 4$.\\

By considering the sum of FITs generated by the BOPs (indicated in red above), we obtain 1 + 15 + 58 = 74.\\

Consider the following tenth degree polynomial generating function:

$$
u_n = 1 - n + n^2 - n^3 + n^4 - n^5 + n^6 - n^7 + n^8 - n^9 + n^{10}
$$

Find the sum of FITs for the BOPs.

\section*{Solution}

We are asked to obtain a polynomial model of degree $k-1$ $(1 \leq k \leq 10)$ that fits the first $k$ elements of $u_n$ $(u_1, u_2, \ldots, u_k)$ exactly, and then evaluate the polynomial model using the element $k+1$, and add the result to the answer.\\

Starting from the fact that if we have $n+1$ points, a polynomial of degree $n$ will cross all points. In practice that is called \textit{over-fitting}, and happens when the model is to complex to fit the data, and when we evaluate the model with other points, is possible that we get a larger error.\\

Then for $k$ points, the polynomial model of degree $k-1$ we have to obtain has the following form:

$$
OP(k,n) = \theta_0 + \theta_1n_1 + \theta_2n^2 + \cdots + \theta_{k-1}n^{k-1}
$$

We need to find the values of $\theta$, and then evaluate $OP(k,n)$ in $k+1$.\\

The equation above can be seen as a \textit{linear model} of the form:

$$
OP(k,n) = \theta_0f_0(n) + \theta_1f_1(n) + \theta_2f_2(n) + \cdots + \theta_{k-1}f_{k-1}(n)
$$

So we can use the \textit{Least Squares} method to obtain the value of $\theta$ and get

$$
\theta = (X^TX)^{-1}X^T y
$$

where $y$ are the values of $u_n$, and $X$ is a matrix with element $X_{i,j}$ representing $f_j(i)$.\\

There are programming languages that have functions that estimates the value of $\theta$. In my case I used \texttt{R}, which have the function \texttt{lm} that estimates the value of $\theta$, and \texttt{predict}, which evaluates a model in a given point.