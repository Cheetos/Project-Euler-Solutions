\addcontentsline{toc}{chapter}{80 - Square root digital expansion}
\chapter*{80 - Square root digital expansion}

\index{square root} 
It is well known that if the square root of a natural number is not an integer, then it is irrational. The decimal expansion of such square roots is infinite without any repeating pattern at all.\\

The square root of two is 1.41421356237309504880..., and the digital sum of the first one hundred decimal digits is 475.\\

For the first one hundred natural numbers, find the total of the digital sums of the first one hundred decimal digits for all the irrational square roots.\\

\section*{Solution}

To solve this problem we used the method of \textit{Square Root by Subtraction} described by Jarvis \cite{square_root}. This method allows us to find the square root of an integer $n$.\\

The algorithm \ref{alg:square_root} receives an integer $n$ and returns the first one hundred decimal digits of $\sqrt{n}$. The value of $p$ is set according of the number of decimal digits we want to calculate. 

\begin{algorithm}[H]
\caption{$squareRoot(n)$}
\begin{algorithmic}
    \STATE $a = 5n$
    \STATE $b=5$
    \STATE $p=10^{101}$
    \WHILE{$b < p$}
        \IF{$a \geq b$}
            \STATE $a \gets a-b$
            \STATE $b \gets b + 10$
        \ELSE
            \STATE $a \gets a \times 100$
            \STATE $b \gets \lfloor b/10 \rfloor \times 100 + 5$
        \ENDIF
    \ENDWHILE
    \STATE $b \gets \lfloor b/100 \rfloor$
    \RETURN b
\end{algorithmic}
\label{alg:square_root}
\end{algorithm}
